\section{Introduction}

The goal is to find the linear model:
\begin{equation}
    y = \beta_0 + \beta_1 x
\end{equation}
such that the sum of squared errors between the predicted values and the actual data is minimized.

\section{Linear Model}

The form of the linear model is:
\begin{equation}
    y_i = \beta_0 + \beta_1 x_i + \epsilon_i
\end{equation}
where $y_i$ is the observed value, $x_i$ is the independent variable, $\beta_0$ is the intercept, $\beta_1$ is the slope, and $\epsilon_i$ is the error term.

We wish to find $\beta_0$ and $\beta_1$ such that the predicted values $\hat{y}_i = \beta_0 + \beta_1 x_i$ minimize the sum of squared errors between $\hat{y}_i$ and the observed values $y_i$.

\section{Design Matrix and Observation Vector}

To make the problem more convenient, we represent it using vectors and matrices.

\subsection{Design Matrix}

Define the design matrix $\mathbf{X}$ as:
\begin{equation}
    \mathbf{X} = \begin{bmatrix}
        1 & 1 \\
        1 & 2 \\
        1 & 3 \\
        1 & 4
    \end{bmatrix}
\end{equation}
The first column contains only 1s, representing the constant term $\beta_0$, and the second column contains the values of the independent variable $x_i$.

\subsection{Observation Vector}

Define the observation vector $\mathbf{y}$ as:
\begin{equation}
    \mathbf{y} = \begin{bmatrix}
        2 \\
        3 \\
        5 \\
        7
    \end{bmatrix}
\end{equation}
This vector contains all the observed values $y_i$.

\subsection{Parameter Vector}

Define the parameter vector:
\begin{equation}
    \boldsymbol{\beta} = \begin{bmatrix} 
        \beta_0 \\ 
        \beta_1 
    \end{bmatrix}.
\end{equation}

\section{Sum of Squared Errors Objective Function}

In regression, our goal is to find the parameters $\boldsymbol{\beta}$ such that the predicted values $\hat{\mathbf{y}} = \mathbf{X} \boldsymbol{\beta}$ are as close as possible to the observed values $\mathbf{y}$, by minimizing the sum of squared errors (SSE):

\begin{equation}
    S(\beta_0, \beta_1) = \sum_{i=1}^n (y_i - \hat{y}_i)^2 = (\mathbf{y} - \mathbf{X} \boldsymbol{\beta})^T (\mathbf{y} - \mathbf{X} \boldsymbol{\beta}).
\end{equation}

\section{Deriving the Normal Equation}

The key idea of least squares is to find $\boldsymbol{\beta}$ such that the residual $\mathbf{y} - \mathbf{X} \boldsymbol{\beta}$ is minimized. Geometrically, this means that the residual should be orthogonal to the column space of the design matrix $\mathbf{X}$, which leads to the normal equation:

\begin{equation}
    \mathbf{X}^T (\mathbf{y} - \mathbf{X} \hat{\boldsymbol{\beta}}) = 0.
\end{equation}

Expanding this:

\begin{equation}
    \mathbf{X}^T \mathbf{y} = \mathbf{X}^T \mathbf{X} \hat{\boldsymbol{\beta}}.
\end{equation}

This is the normal equation, which can be solved to find the least squares estimate $\hat{\boldsymbol{\beta}}$.

\section{Solving the Normal Equation}

Now, let's compute the parts of the normal equation.

\subsection{Compute $\mathbf{X}^T \mathbf{X}$}

\begin{equation}
    \mathbf{X}^T \mathbf{X} = \begin{bmatrix}
        1 & 1 & 1 & 1 \\
        1 & 2 & 3 & 4
    \end{bmatrix}
    \begin{bmatrix}
        1 & 1 \\
        1 & 2 \\
        1 & 3 \\
        1 & 4
    \end{bmatrix}
    = \begin{bmatrix}
        4 & 10 \\
        10 & 30
    \end{bmatrix}.
\end{equation}

\subsection{Compute $\mathbf{X}^T \mathbf{y}$}

\begin{equation}
    \mathbf{X}^T \mathbf{y} = \begin{bmatrix}
        1 & 1 & 1 & 1 \\
        1 & 2 & 3 & 4
    \end{bmatrix}
    \begin{bmatrix}
        2 \\
        3 \\
        5 \\
        7
    \end{bmatrix}
    = \begin{bmatrix}
        17 \\
        50
    \end{bmatrix}.
\end{equation}

\subsection{Solve the Normal Equation}

Now we solve the normal equation:

\begin{equation}
    \begin{bmatrix}
        4 & 10 \\
        10 & 30
    \end{bmatrix} \hat{\boldsymbol{\beta}} = \begin{bmatrix}
        17 \\
        50
    \end{bmatrix}.
\end{equation}

To solve this, we first compute the inverse of $\mathbf{X}^T \mathbf{X}$:

\begin{equation}
    (\mathbf{X}^T \mathbf{X})^{-1} = \frac{1}{20} \begin{bmatrix}
        30 & -10 \\
        -10 & 4
    \end{bmatrix}.
\end{equation}

Next, we compute $\hat{\boldsymbol{\beta}}$:

\begin{equation}
    \hat{\boldsymbol{\beta}} = (\mathbf{X}^T \mathbf{X})^{-1} \mathbf{X}^T \mathbf{y}.
\end{equation}

\begin{equation}
    \hat{\boldsymbol{\beta}} = \frac{1}{20} \begin{bmatrix}
        30 & -10 \\
        -10 & 4
    \end{bmatrix}
    \begin{bmatrix}
        17 \\
        50
    \end{bmatrix}
    = \frac{1}{20} \begin{bmatrix}
        10 \\
        30
    \end{bmatrix}
    = \begin{bmatrix}
        0.5 \\
        1.5
    \end{bmatrix}.
\end{equation}

\section{Least Squares Estimate}

By solving the normal equation, we find $\hat{\beta}_0 = 0.5$ and $\hat{\beta}_1 = 1.5$. Thus, the regression model is:

\begin{equation}
    \hat{y} = 0.5 + 1.5x.
\end{equation}

\section{Conclusion}

Using the projection approach, we see that the least squares estimate is the projection of the observation vector $\mathbf{y}$ onto the space spanned by the columns of the design matrix $\mathbf{X}$. By solving the normal equation, we found the parameters $\hat{\beta}_0 = 0.5$ and $\hat{\beta}_1 = 1.5$, which minimize the sum of squared errors.
