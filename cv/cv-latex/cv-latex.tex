\documentclass[a4paper,12pt]{article}

%----------------------------------------------------------------------------------------
%	FONT
%----------------------------------------------------------------------------------------

% % fontspec allows you to use TTF/OTF fonts directly
% \usepackage{fontspec}
% \defaultfontfeatures{Ligatures=TeX}

% % modified for ShareLaTeX use
% \setmainfont[
% SmallCapsFont = Fontin-SmallCaps.otf,
% BoldFont = Fontin-Bold.otf,
% ItalicFont = Fontin-Italic.otf
% ]
% {Fontin.otf}

%----------------------------------------------------------------------------------------
%	PACKAGES
%----------------------------------------------------------------------------------------
\usepackage{url}
\usepackage{parskip} 	

%other packages for formatting
\RequirePackage{color}
\RequirePackage{graphicx}
\usepackage[usenames,dvipsnames]{xcolor}
\usepackage[scale=0.9]{geometry}

%tabularx environment
\usepackage{tabularx}

%for lists within experience section
\usepackage{enumitem}

% centered version of 'X' col. type
\newcolumntype{C}{>{\centering\arraybackslash}X} 

%to prevent spillover of tabular into next pages
\usepackage{supertabular}
\usepackage{tabularx}
\newlength{\fullcollw}
\setlength{\fullcollw}{0.47\textwidth}

%custom \section
\usepackage{titlesec}				
\usepackage{multicol}
\usepackage{multirow}

%CV Sections inspired by: 
%http://stefano.italians.nl/archives/26
\titleformat{\section}{\Large\scshape\raggedright}{}{0em}{}[\titlerule]
\titlespacing{\section}{0pt}{10pt}{10pt}

%for publications
\usepackage[style=authoryear,sorting=ynt, maxbibnames=2]{biblatex}

%Setup hyperref package, and colours for links
% \usepackage[unicode, draft=false]{hyperref}
\definecolor{linkcolour}{rgb}{0,0.2,0.6}
% \hypersetup{colorlinks,breaklinks,urlcolor=linkcolour,linkcolor=linkcolour}
% \addbibresource{citations.bib}
% \setlength\bibitemsep{1em}

\usepackage[colorlinks = true,
            linkcolor = linkcolour,
            urlcolor  = linkcolour,
            citecolor = linkcolour,
            anchorcolor = linkcolour]{hyperref}

\newcommand{\MYhref}[3][black]{\href{#2}{\color{#1}{#3}}}%

%for social icons
\usepackage{fontawesome5}

%debug page outer frames
%\usepackage{showframe}

%----------------------------------------------------------------------------------------
%	BEGIN DOCUMENT
%----------------------------------------------------------------------------------------
\begin{document}

% non-numbered pages
\pagestyle{empty} 

%----------------------------------------------------------------------------------------
%	TITLE
%----------------------------------------------------------------------------------------

% \begin{tabularx}{\linewidth}{ @{}X X@{} }
% \huge{Your Name}\vspace{2pt} & \hfill \emoji{incoming-envelope} email@email.com \\
% \raisebox{-0.05\height}\faGithub\ username \ | \
% \raisebox{-0.00\height}\faLinkedin\ username \ | \ \raisebox{-0.05\height}\faGlobe \ mysite.com  & \hfill \emoji{calling} number
% \end{tabularx}

\begin{tabularx}{\linewidth}{@{} C @{}}
\Huge{Jinming Ren} \\[7.5pt]
\href{https://github.com/marcobisky}{\raisebox{-0.05\height}\faGithub\ marcobisky} \ $|$ \ 
% \href{https://linkedin.com/in/username}{\raisebox{-0.05\height}\faLinkedin\ username} \ $|$ \ 
\href{https://marcobisky.moe}{\raisebox{-0.05\height}\faGlobe \ marcobisky.moe} \ $|$ \ 
\href{mailto:marcobisky@outlook.com}{\raisebox{-0.05\height}\faEnvelope \ marcobisky@outlook.com} \ $|$ \ 
\href{tel:+86 17882004164}{\raisebox{-0.05\height}\faMobile \ +86 17882004164} \\
\end{tabularx}

%----------------------------------------------------------------------------------------
% EXPERIENCE SECTIONS
%----------------------------------------------------------------------------------------

%Experience
\section{Education}


\textbf{University of Electronic Science and Technology of China (UESTC)}  \hfill Sept 2022 --- Present

\textbf{University of Glasgow, Dual Degree Program} \hfill Sept 2022 --- Present

\begin{itemize}
    \setlength\itemsep{-0.5em}
    \item \textbf{Major}: Electronic and Computer Engineering; \MYhref{https://marcobisky.moe/cv/score.pdf}{GPA: 3.87/4.0}, \MYhref{https://marcobisky.moe/cv/rank.pdf}{Ranking: 2/164 (Top 1\%).} 
    \item \textbf{Relevant Coursework}: Signals and Systems, Stochastic Processes, Artificial Intelligence and Machine Learning, Information Theory, Electrodynamics, Digital Circuit Design, etc.
    \item \textbf{Online Course}: Abstract Algebra, Complex Analysis, Differential Geometry, Control Theory, etc.
\end{itemize}


\section{Research}

\textbf{GAT-based Multi-Task RL for Robust PVT-Aware Analog Design} \hfill Expected Oct.12, 2025

\vspace{-0.5em} \begin{small} \textit{Research Assistant, \MYhref{https://scholar.google.com/citations?user=1NT1jFMAAAAJ&hl=en}{Professor Yun Li}, UESTC} \end{small} \vspace{-0.5em}

\begin{itemize}
    \setlength\itemsep{-0.5em}
    \item Proposed a GAT-based multi-task Reinforcement Learning framework to optimize analog circuits under diverse PVT corners.
    \item Modeled PVT conditions as graph nodes, enabling adaptive attention to corner-specific bottlenecks.
    \item Reduced specific violations by $19\times$ and simulations by 69\% on \texttt{AnalogGym} benchmarks.
\end{itemize}


\MYhref{https://marcobisky.moe/tinyml/}{\textbf{System-level Co-Design of RISCV Accelerators for TinyML at the Edge}} \hfill Ongoing

\vspace{-0.5em} \begin{small} \textit{Research Assistant, \MYhref{https://scholar.google.com/citations?user=1NT1jFMAAAAJ&hl=en}{Professor Yun Li}, UESTC} \end{small} \vspace{-0.5em}

\begin{itemize}
    \setlength\itemsep{-0.5em}
    \item Designing and implementing hardware-accelerated TinyML kernels that are adaptable and efficient for edge computing using \texttt{Chisel}, \texttt{Verilog}, \texttt{Python} and \texttt{C++}.
    \item Exploring a large multi-dimensional design space using automated methods (e.g.  heuristic and evolutionary algorithms) to identify optimal configurations balancing accuracy, energy, and latency.
\end{itemize}


% \MYhref{https://marcobisky.moe/cv/physics-paper.pdf}{\textbf{Generalized Fields and Extension to Higher Dimensions}} \hfill Oct 2023 --- Feb 2024

% \begin{itemize}
%     \setlength\itemsep{-0.5em}
%     \item Investigated generalized natural fields and behaviors in higher dimensions,  Supervised by \href{https://scholar.google.com/citations?user=Amwv33EAAAAJ&hl=en}{Prof. Yidong Liu}.
%     \item  Provided an alternative perspective of understanding the electric field generated by a charged object by extending it to higher dimensions, making some symmetries in electrostatics much more tangible.
% \end{itemize}

\textbf{Movable Antenna (MA) for Anti-jamming} \hfill Feb 2025 --- Jun 2025

\vspace{-0.5em} \begin{small} \textit{Research Assistant, \MYhref{https://scholar.google.com.sg/citations?user=JtHltIoAAAAJ&hl=en-EN}{Professor Weidong Mei}, UESTC} \end{small} \vspace{-0.5em}

\begin{itemize}
    \setlength\itemsep{-0.5em}
    \item Conducted a heuristic investigation into Anti-jamming through stochastic antenna movement.
\end{itemize}

\section{Projects}



\textbf{\MYhref{https://marcobisky.moe/posts/quadrocopter-control/}{Control} and \MYhref{https://marcobisky.moe/cv/drone-cv.pdf}{Computer Vision} for \MYhref{https://marcobisky.moe/cv/drone.pdf}{Autonomous Quadcopter System}} \hfill Feb 2025 --- Jun 2025

\begin{itemize}
    \setlength\itemsep{-0.5em}
    \item Developed an automatic quadrotor aircraft for objection detection, route planning, and closed-loop flight control.
    \item Used \texttt{ROS2} and \texttt{OpenCV} library to implement originally designed computer vision algorithms for real-time landing area detection.
\end{itemize}

\MYhref{https://github.com/Marcobisky/my-riscv}{\textbf{Design and Visualization of a Complete Single-cycle RV32I CPU Core}} \hfill Jan 2025 --- Mar 2025

\begin{itemize}
    \setlength\itemsep{-0.5em}
    \item Designed and simulated an entire RISCV 32-bit CPU from scratch in \texttt{Verilog} for RTL simulation and in \texttt{Digital} Software for working principle visualization.
    \item Supported basic peripherals: GPIOs, IIC, UART, etc.
    \item Implemented a simple boot ROM in assembly, minimal interrupt service for running a Linux kernel.
\end{itemize}


\MYhref{https://github.com/Marcobisky/ame-entropy-source-coding}{\textbf{Adaptive Markov Entropy Source Encoding}} \hfill Oct 2024 --- Nov 2024

\begin{itemize}
    \setlength\itemsep{-0.5em}
    \item Originally-designed the second-order Markov Adaptive Encoding (AME) to perform source coding of \textit{the Game of Thrones} using \texttt{Python} and \texttt{Matlab}.
    \item Evaluated and compared the performance of AME, Huffman and Fano coding.
\end{itemize}


\MYhref{https://github.com/Marcobisky/CNN-for-mbed}{\textbf{CNN for Embedded Systems}} \hfill Feb 2024 --- May 2024

\begin{itemize}
    \setlength\itemsep{-0.5em}
    \item Integrated a convolutional neural network (CNN) into an MCU using \texttt{C} in \texttt{MbedOS}.
    \item Enabled smart fall detection, body temperature monitoring and real-time data visualization for patients.
\end{itemize}





\textbf{Human Voice Recognition Smart Car} \hfill Sept 2023 --- Dec 2023

\begin{itemize}
    \setlength\itemsep{-0.5em}
    \item Designed and implemented a voice-controlled car on STM32F103 using C standard libraries, supporting actions such as moving forwards/backwards, turning/sliding left/right.
    \item Led a 4-member team in the project.
\end{itemize}

\textbf{Digital Door Lock for Dormitory} \hfill Sept 2023 --- Oct 2023

\begin{itemize}
    \setlength\itemsep{-0.5em}
    \item Designed and implemented an embedded digital door lock system in \texttt{C++} on Nucleo L432KC MCU. 
    \item Developed basic functions include manually setting up password, automatically lock for repeated wrong passwords, OLED message displaying, etc.
    \item Led a 3-member team in the project.
\end{itemize}


\textbf{First Place in ``XinTong Cup" Electronic Design Competition} \hfill Sept 2022 --- Oct 2022

\begin{itemize}
    \setlength\itemsep{-0.5em}
    \item Designed and implemented a 8-key music player using register-based development in \texttt{Keil C51} on STC89C52 MCU.
    \item Developed functions includes single note/chord playing, recording, replay and rewind capability, etc.
\end{itemize}



\section{Relevant Skills}
\begin{tabularx}{\linewidth}{@{}l X@{}}
\textbf{IT Skills} &  \normalsize{Latex, Quarto Markdown, Typst, \MYhref{https://www.bilibili.com/video/BV1AterevErt/?spm_id_from=333.1387.homepage.video_card.click&vd_source=42579e22289b6144ba0b2bdcf99834e3}{Manim}, \MYhref{https://github.com/Marcobisky}{Github}, Microsoft Office.}\\
\textbf{Programming} &  \normalsize{\texttt{C/C++}, \texttt{Python}, \texttt{Matlab}, \texttt{Verilog}, \texttt{Chisel}, \texttt{RISCV Assembly}.} \\  
\textbf{Language} &  \normalsize{Native Chinese, Fluent English.} \\
\end{tabularx}


\section{Awards}

\textbf{Top Academic Scholarship of UESTC (Top 5\%)} \hfill Dec 2023, Dec 2024

\textbf{China National Scholarship (Top 3\%)} \hfill Dec 2024

\MYhref{http://www.moe.gov.cn/srcsite/A17/moe_794/moe_628/202408/t20240806_1144389.html}{\textbf{First Prize: 7th National College Art Exhibition and Performance}} \hfill Sept 2024



\vfill
\center{\footnotesize Last updated: \today}

\end{document}